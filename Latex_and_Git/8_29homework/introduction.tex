\section{Introduction}

Laughter is a universal non-verbal vocalization that conveys positive emotions in human interaction \citep {sauter2010cross, scott2014social}.  The social nature of laughter is also evident by the finding that people are up to 30 times more likely to laugh when with others than when alone \citep{provine1989laughing}.

\subsection{Social connectedness and laughter contagion}
The behaviourally contagious phenomenon is strongly mediated by social bonding, especially the level of connectedness between a \textit{laugher} (person who produces the laughter) and a \textit{listener} (person who receives the laughter): people are more likely to laugh in response to a friend's laughter than to a stranger's \citep{smoski2003antiphonal}.

\subsection{Current Research}

The primary aim of the current research was to investigate how social connectedness between the laughers and listeners affects the perception of laughter contagion. Using spontaneous laughter recordings from 17 interconnected individuals (including friends and family members) collected in 2012, we invited this acquaintance group again to the study and listened to each other’s laughter recording (including laughter from themselves) in a laughter perception task. After listening to each laughter recording, they rated the contagiousness and the social connectedness with the laugher. Furthermore, we manipulated an informing effect, that is, whether participants were presented with the laugher’s name before listening to the recording. Based on the relationship between social connectedness, informing condition, and contagion rating, we predicted that:

\begin{enumerate}
    \item  \textit{Hypothesis 1} (H1) — Strength of Social Connectedness: listeners would experience more contagious laughter when they feel more socially connected with the laugher. 
    \item  \textit{Hypothesis 2}  (H2) — Informing of Laughter’s identity: presenting the names of the laughers before participants listen to the recording would result in a stronger contagious effect.
    \item  \textit{Hypothesis 3}  (H3) — Interaction Effect: An interaction effect between social connectedness and informing effects, such that the impact of social connectedness on laughter contagion may depend on prior informing of the laugher’s identity.
\end{enumerate}
