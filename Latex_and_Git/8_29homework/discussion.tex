\section{Discussion}
\subsection{Summary of main result}
In the current study, we investigated social dynamics in the perception of laughter contagion. To answer this primary research question, we examined whether laughter recordings from more socially connected individuals would be rated as more contagious. Results supported the hypothesized main effect of social connectedness: without any additional context, when people listened to the laughter from more familiar or closer individuals, they experienced a stronger contagious effect. This result aligns with the previous research suggesting that social bonding mediates laughter perception \citep{smoski2003antiphonal}. Importantly, our study extends this line of research by examining a variety of interpersonal relationships (e.g., friends, couples, relatives), contributing to a broader understanding of how social bonding influences the contagiousness of laughter.

We also found a significant informing effect consistent with Hypothesis 2: when participants were given the laugher’s name before listening to the recording, participants perceived the laughter as more contagious than when no name was given. In addition, the significant interaction effect (Hypothesis 3) found that the social connectedness effect was less pronounced in the informed condition. When no information regarding the laughter’s identity was presented, social connectedness showed a linear relationship with the contagion ratings; when informing was applied, contagion ratings were slightly higher overall. To further examine the social effect, in the exploratory analysis, we found that self-laugher was perceived to be more contagious than others’ laughter, regardless of whether participants were informed of the laugher’s identity.

\subsection{Conclusion}
Our study provides new insights into how social connectedness modulates the perception of contagious laughter. We found a social asymmetry underlying the contagion perception and the emotion transfer process. The sense of familiarity created by the mere exposure of the laugher’s name could enhance the contagion perception. In addition, we presented the first evidence that individuals perceive self-laughter as more contagious than others’ laughter, which could also be due to familiarity with oneself. Together, our study lays the ground for the potential of using laughter as a valuable proxy to elucidate socio-emotion dynamics.

Lastly, we revealed individual differences in the perception of laughter contagion. People with impoverished social bonding still experience a comparable level of laughter contagion. Individual differences in Frequency, Liking, and Usage of laughter measured from LPPQ are also closely associated with the contagion rating, which establishes the applicability of the Laughter Production and Perception Questionnaire (LPPQ) to understand the considerable variations in human laughing behaviors.